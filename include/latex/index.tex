This folder contains the header files of functions and classes which are used in the different case folders.

We define 6 modules (groups) in order to document better (with doxygen) the content of these files

\section*{Matrices}

It groups all the classes defined in the \hyperlink{matrix_8hpp}{matrix.\+hpp} file. The matrices that we are talking about are the ones obtained by the discretization of the equations.

First of all a virtual class (the Abstract Matrix class) it\textquotesingle{}s declared which is the prototype for all the derived class whcich inherit from it.

The derived classes are\+:

Matrix A is the mass velocity matrix for the Darcy system;

Matrix B is the saddle point matrix of the Darcy system;

Matrix C is the mass matrix obtained in the transport equation with the finite volume method;

Matrix F\+\_\+piu and F\+\_\+meno are the ones that define the upwind scheme;

Matrix R is the reactive matrix used in the case5\+\_\+2reagents for defining the reactive part of the equation.

\section*{Parameters}

It groups all the classes defined in the \hyperlink{parameters_8hpp_source}{parameters.\+hpp} file.

These classes are used to initialize all the data that are used in order to define the different problems.

Their constructors take as input a data.\+pot file and through Get\+Pot they store the data in different variables.

\section*{Darcy Functions}

It groups all the functions in the \hyperlink{darcy_8hpp_source}{darcy.\+hpp} file.

These functions are the ones that are used to define and solve the Darcy Problem.

\section*{Darcy Output}

It groups all the functions in the darcy\+\_\+output.\+hpp file.

These functions print on csv files the solution given by the code and the exact ones given by a suitable muparser function.

There is also a function that is used to print the error of the convergence analysis in order to see the convergence order of the scheme adopted.

\section*{Output Function}

It groups all the function in the \hyperlink{output_8hpp_source}{output.\+hpp} file.

These functions print on csv files the solution of temporal and spatial problems such as transport and reaction ones.

\section*{Mu\+Parser}

It contains the \hyperlink{classmuparser__fun}{muparser\+\_\+fun} class which is defined in the \hyperlink{muparser__fun_8hpp_source}{muparser\+\_\+fun.\+hpp} file.

This class permit us to read and use functions which are given as handle function in the data.\+pot file. 